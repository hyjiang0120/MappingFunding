\chapter{Introduction}
\linenumbers 
\lettrine[lines=1]{I}{n}
recent years, humankind has faced many daunting challenges, including  military conflicts, rising sea levels gradually making some cities disappear, pandemics, noncommunicable diseases, dwindling resources, and so forth, and are also working hard to iterate technology. Countries have reached a record high of almost US \$1.7 trillion in spending on technological innovation and scientific research \citep{RN16}. \\


Recently, research remains at the forefront of understanding and addressing these complex problems. The foundation of any successful research project relies on innovative ideas and meticulous execution \citep{neema2021research}. Conducting research, especially in fields like wet laboratory science or large-scale epidemiological studies, demands substantial funding to cover tangible and intangible costs \citep{schembri2018wasp}. Therefore, even though the process of applying for funding is by no means an easy task and is considerably daunting and time-consuming, with EU research programs, for example, EU research (such as Horizon 2020), application success rates are only around 15\% \citep{schembri2018wasp}, researchers generally opt to seek research funding to ensure the smooth operation of their research projects. Encouragingly, governments, universities, and nonprofit organizations recognize the pivotal role of research and development (R\&D) in driving economic growth, job creation, national security, environmental protection, and knowledge expansion \citep{sargent2017global}. \\

In 2020, global R\&D expenditures reached \$2.352 trillion \citep{sargent2017global}. The 10 largest R\&D-funding countries of 2020 accounted for \$1.999 trillion in R\&D expenditures \citep{sargent2017global}. Such substantial investments highlight the commitment to foster research and innovation. In fact, securing adequate funding is crucial to fueling the relentless pursuit of scientific breakthroughs. For instance, Gush et al.'s findings indicate that funding is correlated with a 6-15\% increase in publications and a 22-26\% increase in citation-weighted papers for research teams \citep{gush2018effect}.\\


In 2021, The UK government’s net expenditure on research and development (R\&D), excluding EU contributions, remained at \pounds14.0 billion. Within the UK, research funding takes two primary forms: commercial and non-commercial, with the latter dominating the landscape. Non-commercial funding sources encompass research charities, national academies, various government departments, and the United Kingdom Research and Innovation (UKRI).\\

Among these organizations, UKRI is UK's most significant public funder of research and innovation, principally funded through the Science Budget by the Department for Business, Energy and Industrial Strategy (BEIS). According to UKRI Annual Report and Accounts 2021-22, they invest more than £8 billion annually to advance our understanding of people and the world around us and deliver benefits for society, the economy, and the environment. \\

From everyday conversations and the information we gather from online news, we might form a rough perception that the field of science and technology tends to secure a higher frequency of funding for projects. However, these notions are not substantiated and are essentially intuitive assumptions. Moreover, as time progresses, the emergence of novel discoveries continuously challenges or validates our existing beliefs, potentially reshaping the landscape of future funding allocation. This inevitably prompts the question: Can discoveries wield a fresh influence on funding arrangements?\\

Hence, through a comprehensive analysis of investment patterns, I can uncover the focal points of institutions and gain insights into the developmental trajectories of various specialized domains. Furthermore, this analysis provides a more precise understanding of the rise and fall of various domains and their specific manifestations in terms of funding allocation. Understanding research funding trends carries immense significance for researchers, policymakers, and funding agencies. By unraveling the ever-changing landscape of research investment, people can identify shifts in research priorities and potential areas of innovation. Such insights will empower us to address contemporary challenges effectively and adapt research strategies for future endeavors.\\

Notably, I seek to address the following key research questions:

\begin{enumerate}
  \item Evolution of Research Funding Priorities: How have research funding priorities evolved across different disciplines and industries? 
  \item Emerging Research Areas: What emerging research areas have gained prominence recently, and how do they align with societal needs and technological advancements?
  \item Drivers of Fluctuations: What are the driving factors behind fluctuations in research attention to specific themes in different periods?\\
  \item What is the funding trend for ecology and evolution?\\
 
\end{enumerate}

I will employ AI-driven methods to process and visualize the data, revealing patterns and connections between funding allocation and significant temporal factors. Monitoring and comprehending shifts in research investment will empower me to effectively tackle current challenges and tailor research strategies for upcoming pursuits.\\