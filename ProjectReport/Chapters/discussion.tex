\chapter{Discussion}
\lettrine[lines=1]{V}{arious}
factors guide the decisions different countries make regarding science funding allocation. The anticipated outcomes of this project's research are evident in many cases. Many disciplines within the natural and applied sciences rely heavily on funding, with substantial investments required for experimental materials and equipment. Moreover, the results generated within these two domains are more readily applicable, yielding greater economic advantages.\\

In contrast, formal sciences and many social science disciplines lean towards advancing our fundamental understanding of nature rather than providing information directly applicable in practical contexts \citep{shaw2022revisiting}. This propensity often results in fewer institutions funding research due to limited immediate tangible benefits. Apart from economic considerations, funding decisions are increasingly subject to external pressures. The demand for societal relevance may influence expert panel perceptions of research quality. Some researchers even express concerns that institutional assessments such as the "Research Excellence Framework" in the UK could adopt criteria that ultimately become ineffective substitutes for gauging research quality \citep{meirmans2019science}.\\

In this context, fields such as public health, computer science, ecology, fluid dynamics, biology, and pharmaceutical science have emerged as popular and well-funded domains, and the factors driving this sustained support are multifaceted. Firstly, in the post-pandemic era, the prominence of public health, pharmaceutical science, and biology can be attributed to their widespread societal demand and profound impact. Particularly, research in public health encompasses human well-being and disease prevention, directly influencing overall societal health and quality of life. Simultaneously, against the backdrop of escalating environmental pollution and climate change concerns, research in environmental science, ecology, materials science, and sustainable energy holds crucial significance for environmental preservation and ecological equilibrium. Additionally, computer science plays a pivotal role in digitization, exerting substantial influence on technological innovation, information technology, and data analytics.\\

Moreover, the research and applications within these fields directly address numerous real-world challenges. Especially in the current era of advanced artificial intelligence, computer science is a foundational discipline underpinning the profound development of emerging fields. As a fundamental scientific field, the continuous funding support for fluid dynamics further underscores the UK's commitment to international competitiveness and influence.\\

In addition, the research has also yielded some unforeseen outcomes. For instance, the substantial investments made by UKRI in the environmental and renewable energy sectors during 2016-2018, as well as the stark contrast between the increasing number of projects receiving investments in the applied science domain and the diminishing investments in certain specific subfields. The heightened focus on environmental concerns during this period could be attributed to the historical milestones of 2016, marked by record-breaking global temperatures, diminished polar ice, rising sea levels, and escalating ocean heat content. The persistence of extreme weather events and climatic conditions into 2017 drew global attention, potentially influencing UKRI's decision-making processes.\\

Furthermore, the aggregated nature of investments in the applied science domain might reflect UKRI's inclination towards channels that have been previously funded. Continuous investments in domains such as Medical Technology, Agricultural Science, and Immunology may signify positive developments in these fields, suggesting ongoing advancements or the likelihood of breakthrough progress in the near future.\\


Furthermore, with the rapid development of artificial intelligence and platforms like ChatGPT in recent years, the pervasive impact of AI across diverse sectors of the internet has garnered widespread attention. The dynamic evolution of AI technologies and modeling techniques is noteworthy. It is reasonable to anticipate a potential increase in funding or the number of projects in AI, modeling, and related fields.\\


\section*{Limitation}
Certainly, given the incompleteness of data for the years 2022-2024, it is essential to acknowledge that the current analysis comes with limitations and potential biases stemming from pending projects still in the application phase. However, from my perspective, the emphasis on addressing environmental pollution and promoting public health has been consistently evident. As a result, fields related to environmental concerns and various branches of biology remain focal points for funding allocation.\\

\section*{Conclusions}
In the intricate landscape of science funding allocation, the decisions made by different countries are guided by a complex interplay of factors. This study unveiled various trends and considerations that shape funding distribution across different scientific domains. Notably, funding allocation is multifaceted, influenced by societal demands, economic advantages, and research relevance. While applied and natural sciences often garner substantial funding due to their practical applications and economic benefits, formal sciences and certain social science disciplines focus on advancing fundamental understanding rather than immediate practicality.

Public health, computer science, ecology, fluid dynamics, and pharmaceutical science have emerged as well-funded and impactful domains. The pandemic has underscored the importance of public health and pharmaceutical research, while environmental concerns and technological advancements have driven investments in ecology, sustainable energy, and computer science.


In conclusion, science funding allocation is a dynamic process influenced by many factors. The trends identified in this study reflect a balance between societal needs, economic considerations, and research objectives. With its ever-changing challenges and opportunities, the evolving scientific landscape will continue to shape funding priorities and drive innovation in the years to come.


\section*{Data and Code Availability}
The original dataset for this project can be obtained from https://www.dropbox.com/scl/fo/x85cc5gxk8uxu5x1c5lpz/h?rlkey=5ryaw51zilehxhisyah7prt4u&dl=0. The corresponding source code is available at https://github.com/hyjiang0120/MappingFunding.

\nolinenumbers