\begin{abstract}


\linenumbers 
Countries worldwide have always paid continuous attention to and supported scientific research. Based on our daily experience, it seems that fields such as biology or AI have received a lot of attention recently, but this lacks research support. So what does the current scientific landscape look like? And how did it arrive at the current pattern?

I take the UK Research and Innovation (UKRI) as a case study to analyze investment directions and structures, aiming to yield profound insights into the funding landscape and its development process. Employing a data-driven approach, I leverage machine learning techniques, particularly the Mallet Latent Dirichlet Allocation (LDA).\\

The ultimate discovery underscores a pronounced tendency to allocate funds towards applied and natural sciences, reflecting their paramount significance within the research agenda. During the 20 years of analysis, in each three-year period, more than 80\% of the invested projects belong to applied and natural science. 2022-2024, according to incomplete statistics, the two companies received 6,785,877,536 pounds of funding, accounting for 92.15\% of the total expenditure funding amount. Furthermore, ecology has consistently received significant attention and continuous funding over the years. During the period from 2016 to 2018, it secured a total funding of £108,990,748. This amount positioned ecology as the second highest-funded field within natural sciences. Nevertheless, the intricacies of these domains have given rise to many specialized orientations. Alternatively, while receiving fewer and comparatively modest allocations in the social sciences domain, public policy occupies a dominant position. Apart from the period between 2010 and 2012, during which business projects also received funding, funds were exclusively allocated to Public Policy within the realm of social sciences in all other years.\\

By analyzing the dynamic funding allocation in UKRI, I unveil the ever-evolving priorities and prospects within the funding landscape, revealing the history of scientific development and possible future development trends.\\







\end{abstract}
